\documentclass{article}
\begin{document}
\title{A CONCEPT PAPER FOR CHECKING PROBABILISTIC DENIABILITY}
\author{CS EVE Group 207 }
\maketitle
\section{Executive Summary(Abstract)}
We use the probabilistic deniability to analyze the Crowds system for anonymous
Web browsing. This case study demonstrates how probabilistic deniability
techniques can be used to formally analyze security properties of a peer-to-peer
group communication system based on random message routing among members. The
behavior of group members and the adversary is modeled as a discrete-time Markov
chain, and the desired security properties are expressed as PCTL formulas.ur main result is a demonstration of how certain
forms of probabilistic anonymity degrade when group size increases or random
routing paths are rebuilt, assuming that the corrupt group
members are able to identify and/or correlate multiple routing paths
originating from the same sender.
\section{Introduction}
Formal analysis of security protocols is a well-established field. Model
checking and theorem proving techniques have been extensively used to
analyze secrecy, authentication and other security properties systems that employ
cryptographic primitives such as public-key encryption, digital signatures, etc.
Conventional formal analysis of security is mainly concerned with security
 against the so called Dolev-Yao attacks Many proposed
systems for anonymous communication aim to provide strong, non-probabilistic anonymity guarantees. Non-probabilistic anonymity systems are
amenable to formal analysis in the same non-deterministic Dolev-Yao model as used
for verification of secrecy and authentication protocols.In this
project, we use probabilistic model checking to analyze anonymity
properties of a gossip-based system. Such systems fundamentally rely on
probabilistic

message routing to guarantee anonymity. The first flaw was identified by the
authors of

Crowds [RR98] and analyzed by Makhi [Mal01] and Wright et al.
[WALS02] Previous research on probabilistic formal models for security
focused on
\begin{enumerate}
\item Probabilistic characterization of non-interference, and
\item Process formalisms that aim to faithfully model probabilistic
properties of cryptographic primitives. This paper attempts to directly model
and analyze security properties based on discrete probabilities, as
opposed to asymptotic probabilities in the conventional cryptographic sense. Our
analysis method is applicable to other probabilistic anonymity systems
such as free net and onion routing.
\end{enumerate}
\section{Problem Statement}
The problem this project will tackle is to maintain the privacy of
communicated data against passive eavesdroppers (A secret listener to private
conversations). The key solution to the problem is by constructing a
sender-deniable public key encryption scheme based on any trapdoor permutation.
However, our scheme falls short of achieving the desired level of deniability.
\section{Objectives}
\subsection{General Objectives}
 To analyze the anonymity properties of a crowds system using a probabilistic model checker (PRISM).
\subsection{Specific Objectives}
\begin{enumerate}
\item To prevent corrupt crowd members from linking multiple paths and using this information to infer the initiator’s identity, the Crowds paper suggests that paths should be static.
\item To formally specify properties of the crowd system to be checked using a temporal probabilistic PCTL(Probabilistic Computation Tree Logic) formulas.
\item To hide each user’s communication by routing them randomly within a crowd of similar users.
\end{enumerate}

\section{Scope}
 This case study demonstrates how probabilistic model checking techniques can be used to formally analyze security
properties of a peer-to-peer group communication system based on random message routing among members.
\section{Research Significance}
Our main result is a demonstration of how certain forms of probabilistic anonymity degrade when group size increases or random routing paths are rebuilt.
\section{Literature Review}

\begin{enumerate}
\item \textbf{{Markov Chain Model Checking.}}\newline
 We model the probabilistic behavior of a peer-to-peer communication system as a discrete-time Markov chain (DTMC), which is a standard approach in probabilistic verification
\item \textbf{{ PRISM Model Checker}}\newline
 The automated analyses described in this paper were performed using PRISM, a
probabilistic model checker developed by Kwiatkowska et al.. The tool supports both discrete- and continuous-time Markov chains, and Markov decision processes. We model probabilistic peer-to-peer communication systems such as Crowds simply as discrete-time Markov chains, and formalize their properties in PCTL(Probabilistic Computation Tree Logic).
\item\textbf{{ Anonymity properties of Crowds.}}\newline
 The Crowds paper describes several degrees of anonymity that may be provided by a communication system. Without using anonymizing techniques, none of the following properties are guaranteed on the Web since browser requests contain information about their source and destination in the clear i.e. Beyond suspicion, Probable innocence, Possible innocence.
\end{enumerate}
\section{Conclusion}
Probabilistic deniability is a well-established technique for verification of hardware and concurrent protocols. The main contribution of this paper is to hint how it can be applied to the analysis of security properties based on discrete probabilities. We analyzed anonymity properties of the Crowds system, a “real-world” protocol for anonymous Web browsing. Anonymity in Crowds is based on constructing a random routing path to the destination through a group of members, some of whom may be corrupt. The path construction protocol is purely probabilistic, therefore, we modeled it as a discrete time Markov chain, without introducing non-determinism and thus avoiding the need for Markov decision processes.
 \begin{thebibliography}{1}

\bibitem{notes} R. Alur and T. Henzinger {\em Reactive modules. In Proc. 11th Annual
IEEE Symposium on Logic in Computer Science (LICS), pages 207–
218} 1996.

\bibitem{impj}  C. Baier {\em On algorithmic verification methods for probabilistic
systems} 1998. Fakult¨at f¨ur Mathematik and Informatik,
Universit¨at Mannheim.

\bibitem A. Bianco and L. de Alfaro.{\em Model checking of probabilistic and
nondeterministic systems. In Proc. Foundations of Software Technology
and Theoretical Computer Science (FST and TCS), volume 1026 of
LNCS, pages 499–513. Springer-Verlag} 1995
  \end{thebibliography}

\end{document}
