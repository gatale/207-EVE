\documentclass[10pt]{article}
\usepackage{makeidx}
\usepackage{multirow}
\usepackage{multicol}
\usepackage[dvipsnames,svgnames,table]{xcolor}
\usepackage{graphicx}
\usepackage{epstopdf}
\usepackage{ulem}
\usepackage{hyperref}
\usepackage{amsmath}
\usepackage{amssymb}
\author{Admin}
\title{}
\usepackage[paperwidth=612pt,paperheight=792pt,top=72pt,right=72pt,bottom=72pt,left=72pt]{geometry}

\makeatletter
	\newenvironment{indentation}[3]%
	{\par\setlength{\parindent}{#3}
	\setlength{\leftmargin}{#1}       \setlength{\rightmargin}{#2}%
	\advance\linewidth -\leftmargin       \advance\linewidth -\rightmargin%
	\advance\@totalleftmargin\leftmargin  \@setpar{{\@@par}}%
	\parshape 1\@totalleftmargin \linewidth\ignorespaces}{\par}%
\makeatother 

% new LaTeX commands


\begin{document}


\begin{center}
{\large A CONCEPT PAPER FOR CHECKING PROBABILISTIC DENIABILITY}
\end{center}

{\raggedright
\textbf{Abstract}
\\
we use the probabilistic deniability to analyze the Crowds system for anonymous
Web browsing. This case study demonstrates how probabilistic deniability
techniques can be used to formally analyze security properties of a peer-to-peer
group communication system based on random message routing among members. The
behavior of group members and the adversary is modeled as a discrete-time Markov
chain, and the desired security properties are expressed as PCTL formulas.{\large
 } \textbf{Our main result is a demonstration of}\textbf{ }\textbf{how certain
forms of probabilistic anonymity degrade when group size increases or random
routing paths are rebuilt, assuming that the corrupt group}\textbf{
}\textbf{members are able to identify and/or correlate multiple routing paths
originating from the same sender.}
}

\begin{enumerate}
	\item \textbf{Introduction}
\\
Formal analysis of security protocols is a well-established field. Model
checking{\large  }and theorem proving techniques have been extensively used to
analyze secrecy, authentication and other security properties systems that employ
cryptographic primitives such as public-key encryption, digital signatures, etc.
Conventional formal analysis of security is mainly concerned with security{\large
 }against the so called \textit{Dolev-Yao attacks}.{\large  }Many proposed
systems for anonymous communication aim to provide strong,
\\
non-probabilistic anonymity guarantees. Non-probabilistic anonymity systems are
amenable to formal analysis in the same non-deterministic Dolev-Yao model as used
for verification of secrecy and authentication protocols.{\large  }In this
project, we use \textit{probabilistic }model checking to analyze anonymity
properties of a gossip-based system. Such systems fundamentally rely on
probabilistic
\\
message routing to guarantee anonymity. The first flaw was identified by the
authors of
\\
Crowds [RR98] and analyzed by Makhi [Mal01] and{\large  }Wright \textit{et al.
}[WALS02].{\large  }Previous research on probabilistic formal models for security
focused on
\end{enumerate}

{\raggedright
(i){\large  }Probabilistic characterization of non-interference, and
}

{\raggedright
(ii){\large  }Process formalisms that aim to faithfully model probabilistic
properties of cryptographic primitives. This paper attempts to directly model
and{\large  }analyze security properties based on discrete probabilities, as
opposed to asymptotic probabilities in the conventional cryptographic sense. Our
analysis method{\large  }is applicable to other probabilistic anonymity systems
such as free net{\large  }and onion routing.
}

\begin{enumerate}
	\item \textbf{{\large Problem statement}}
\end{enumerate}

{\raggedright
{\large The problem this project will tackle is to maintain the privacy of
communicated data against passive eavesdroppers (A secret listener to private
conversations). The key solution to the problem is by constructing a
sender-deniable public key encryption scheme based on any trapdoor permutation.
However, our scheme falls short of achieving the desired level of deniability. }
}
\begin{enumerate}
	\item \textbf{{\large General Objective}}
	{\raggedright
{\large To analyze the anonymity properties of a crowds system using a probabilistic model checker (PRISM). }
}
\end{enumerate}
\begin{enumerate}
	item\textbf{{\large Conclusion}}
\end {enumerate}
{\raggedright
{\large 7.	Conclusions
Probabilistic deniability is a well-established technique for verification of hardware and concurrent protocols. The main contribution of this paper is to hint how it can be applied to the analysis of security properties based on discrete probabilities. We analyzed anonymity properties of the Crowds system, a “real-world” protocol for anonymous Web browsing. Anonymity in Crowds is based on constructing a random routing path to the destination through a group of members, some of whom may be corrupt. The path construction protocol is purely probabilistic, therefore, we modeled it as a discrete time Markov chain, without introducing non-determinism and thus avoiding the need for Markov decision processes.
}
}
{\large  }
\textbf{Word-to-LaTeX TRIAL VERSION LIMITATION:}\textit{ A few characters will be randomly misplaced in every paragraph starting from here.}


\end{document}
